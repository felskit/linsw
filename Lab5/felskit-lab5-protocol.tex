\documentclass[10pt,a4paper]{article}
\usepackage[T1]{fontenc}
\usepackage[utf8]{inputenc}
\usepackage{graphicx}
\usepackage{tabularx}
\usepackage{helvet}
\usepackage{hyperref}
\usepackage[a4paper,margin=1in]{geometry}
\usepackage[polish]{babel}
\usepackage{float}
\usepackage{xcolor}
\usepackage{listings}
\usepackage{inconsolata}
\lstdefinestyle{bash}{
  basicstyle=\ttfamily\color{black},
  captionpos=b,
  commentstyle=\color{teal},
  keywordstyle=\color{blue},
  stringstyle=\color{olive},
  language=bash,
  moredelim=**[is][\color{blue}]{^}{^},
  showstringspaces=false,
  columns=fullflexible
}

\renewcommand\familydefault{\sfdefault}

\title{Linux w systemach wbudowanych -- Laboratorium 5}
\author{Tymon Felski}

\begin{document}

\makeatletter
\begin{center}
	\LARGE{\@title}\\
	\vspace{.4cm}
	\Large{\@author}\\
	\vspace{.2cm}
	\large{\today}
\end{center}
\makeatother

\section{Treść zadania}
Podczas piątego laboratorium należało przenieść projekt stworzony w ramach czwartego laboratorium do środowiska innego niż Buildroot. Cel laboratorium był taki sam jak poprzednio, czyli przygotowanie przy pomocy płytki Raspberry Pi urządzenia wyposażonego w złożony interfejs użytkownika, w którym:
\begin{itemize}
	\item przyciski i diody LED powinny być użyte do podstawowej obsługi urządzenia,
	\item interfejs WWW lub inny interfejs sieciowy powinien być użyty do bardziej zaawansowanych funkcji,
	\item przetwarzany powinien być dźwięk lub obraz.
\end{itemize}

\section{Odtwarzanie projektu z załączonego archiwum}
Należy wykonać kroki opisane poniżej.

\section{Opis rozwiązania}
\subsection{Przygotowanie}
Wybór środowiska padł na \texttt{LEDE Project}. Pobrano obraz systemu w wersji \texttt{17.01}. Znajduje się on w dołączonym archiwum pod nazwą \texttt{lede.img.gz}.\\[\baselineskip]
Pierwszym krokiem było wgranie systemu na kartę pamięci płytki. W tym celu uruchomiono płytkę w trybie ratunkowym (przerywając bootowanie i wpisując \texttt{run rescue}), a następnie wykonano polecenie
\begin{lstlisting}[style=bash, commentstyle=\color{black}]
# ssh felskit@192.148.143.XXX "cat /malina/felskit/lede.img.gz" \
> | gunzip -c | dd of=/dev/mmcblk0 bs=4M
\end{lstlisting}
Po rozpakowaniu i wgraniu obrazu oraz ponownym uruchomieniu płytki można było przystąpić do konfiguracji systemu. Niezbędne było skonfigurowanie połączenia sieciowego.
\begin{lstlisting}[style=bash, commentstyle=\color{black}]
# udhcpc -i br-lan
# uci set network.lan.proto=dhcp
# uci commit
# /etc/init.d/network restart
\end{lstlisting}
To zapewniło działające połączenie z siecią. Kolejnym krokiem była aktualizacja listy paczek dostępnych przez \texttt{opkg}.
\begin{lstlisting}[style=bash, commentstyle=\color{black}]
# opkg update
\end{lstlisting}
Aplikacja przygotowana na poprzednim laboratorium korzysta z \texttt{mpd} oraz \texttt{mpc} przy odtwarzaniu muzyki, więc należało je zainstalować poleceniami
\begin{lstlisting}[style=bash, commentstyle=\color{black}]
# opkg install mpd
# opkg install mpc
\end{lstlisting}
Domyślnie instalującym się \texttt{mpd} jest \texttt{mpd-mini}, jednak w zupełności wystarczy do zaspokojenia potrzeb aplikacji. Niezbędny był też \texttt{Python} w wersji 3 i odpowiadający mu package manager.
\begin{lstlisting}[style=bash, commentstyle=\color{black}]
# opkg install python3
# opkg install python3-pip
\end{lstlisting}
\newpage\noindent
Aby uniknąć komplikacji przy instalacji pozostałych pakietów, zaktualizowano \texttt{setuptools}.
\begin{lstlisting}[style=bash, commentstyle=\color{black}]
# pip3 install --upgrade setuptools
\end{lstlisting}
Do działania przygotowanej aplikacji niezbędy jest moduł \texttt{flask}, jednak jeden z pakietów, od których on zależy, wymaga modułu \texttt{six}. Wobec tego wykonano polecenia
\begin{lstlisting}[style=bash, commentstyle=\color{black}]
# pip3 install six
# pip3 install flask
\end{lstlisting}
Aplikacja korzysta także z interfejsu \texttt{GPIO} do zarządzania diodami i przyciskami na płytce Raspberry Pi, więc wymagany jest moduł, który spełni to wymaganie. Niestety instalacja wykorzystywanego podczas poprzedniego laboratorium modułu \texttt{RPi.GPIO} nie jest możliwa pod używanym obrazem \texttt{LEDE}, chociażby ze względu na brak \texttt{ccache}. Wybrano jego uproszczoną alternatywę - \texttt{gpio}.
\begin{lstlisting}[style=bash, commentstyle=\color{black}]
# pip3 install gpio
\end{lstlisting}
Tak przygotowany system był w stanie uruchomić aplikację, jednak należało w niej najpierw wprowadzić kilka modyfikacji, które opisane są w następnym punkcie.

\subsection{Aplikacja}
Stworzona w ramach laboratorium aplikacja to internetowe radio napisane w Pythonie, którym można sterować przy pomocy interfejsu \texttt{GPIO} oraz interfejsu webowego. Ze względu na zmianę modułu obsługującego interfejs \texttt{GPIO}, aplikacja musiała zostać nieznacznie zmodyfikowana.\\[\baselineskip]
Pierwszym krokiem była zmiana importowanego modułu.
\begin{lstlisting}[style=bash, language=python]
- import RPi.GPIO as GPIO
+ import gpio as GPIO
\end{lstlisting}
Ponadto, należało usunąć linijkę konfigurującą tryb działania interfejsu.
\begin{lstlisting}[style=bash, language=python]
- GPIO.setmode(GPIO.BCM)
\end{lstlisting}
Konieczna była zmiana sposobu inicjacji przycisków.
\begin{lstlisting}[style=bash, language=python]
- GPIO.setup(button, GPIO.IN, pull_up_down = GPIO.PUD_UP)
+ GPIO.setup(button, GPIO.IN, initial=True)
\end{lstlisting}
Niezbędna okazała się także zmiana odniesień do stanu przysków i diód poprzez \texttt{GPIO.HIGH} oraz \texttt{GPIO.LOW} na odpowiednio \texttt{1} oraz \texttt{0}. Ostatnim krokiem było odpowiednie wywołanie czyszczenia stanu diód, które polega już na jednorazowym wywołaniu \texttt{GPIO.cleanup()}.
\begin{lstlisting}[style=bash, language=python]
for led in [17, 18, 23, 24]:
    GPIO.cleanup(led)
\end{lstlisting}
Tak zmodyfikowana aplikacja mogła zostać bez problemu uruchomiona na wcześniej skonfigurowanym obrazie systemu dostarczonym przez \texttt{LEDE Project}. Przy pomocy \texttt{ssh} folder z radiem znajdujący się w dostarczonym archiwum został skopiowany z komputera laboratoryjnego na kartę pamięci płytki.
\begin{lstlisting}[style=bash]
$ scp -r radio/ root@192.168.143.XXX:/radio
\end{lstlisting}

\end{document}
