\documentclass[10pt,a4paper]{article}
\usepackage[T1]{fontenc}
\usepackage[utf8]{inputenc}
\usepackage{graphicx}
\usepackage{tabularx}
\usepackage{helvet}
\usepackage{hyperref}
\usepackage[a4paper,margin=1in]{geometry}
\usepackage[polish]{babel}
\usepackage{float}
\usepackage{xcolor}
\usepackage{listings}
\usepackage{inconsolata}
\lstdefinestyle{bash}{
  basicstyle=\ttfamily\color{black},
  captionpos=b,
  commentstyle=\color{teal},
  keywordstyle=\color{blue},
  stringstyle=\color{olive},
  language=bash,
  moredelim=**[is][\color{blue}]{^}{^},
  showstringspaces=false,
  columns=fullflexible
}

\renewcommand\familydefault{\sfdefault}

\title{Linux w systemach wbudowanych -- Laboratorium 2}
\author{Tymon Felski}

\begin{document}

\makeatletter
\begin{center}
	\LARGE{\@title}\\
	\vspace{.4cm}
	\Large{\@author}\\
	\vspace{.2cm}
	\large{\today}
\end{center}
\makeatother

\section{Treść zadania}
Podczas drugiego laboratorium należało przygotować aplikację w języku C, która powinna:
\begin{itemize}
	\item obsługiwać przyciski i diody LED płytki Raspberry Pi za pomocą interfejsu GPIO,
	\item reagować na zmiany stanu przycisków za pomocą przerwań (bez oczekiwania aktywnego),
	\item realizować funkcjonalność ustaloną przez studenta.
\end{itemize}
Celem ćwiczenia było również przygotowanie pakietu Buildroota ze stworzoną aplikacją i zdalne przetestowanie jej z komputera laboratoryjnego za pomocą debuggera \texttt{gdb}.

\section{Odtwarzanie projektu z załączonego archiwum}
Dostarczone archiwum należy umieścić w dowolnym miejscu na dysku i rozpakować. Uruchomienie skryptu \texttt{install.sh}, znajdującego się w środku archiwum, spowoduje odtworzenie konfiguracji środowiska z laboratorium. Jako parametr należy podać ścieżkę do katalogu z rozpakowanymi plikami Buildroota.\\[\baselineskip]
Skrypt utworzy katalog \texttt{felskit-lab2/} równoległy do katalogu przekazanego jako argument. Następnie do obu z nich zostaną przekopiowane odpowiednie pliki i katalogi. Ponadto, skrypt zastosuje domyślną konfigurację dla płytki Raspberry Pi i właściwie spatchuje pliki konfiguracyjne Buildroota. Na koniec skrypt zapyta czy rozpocząć kompilację jądra systemu.

\section{Opis rozwiązania}
\subsection{Przygotowanie}
W celu przygotowania środowiska Buildroot, ustawiono następujące opcje:
\begin{enumerate}
	\item \textit{System configuration} $\rightarrow$ \textit{Port to run a getty (login prompt) on} na \textit{ttyAMA0}
	\item \textit{Build options} $\rightarrow$ \textit{Mirrors and Download locations} $\rightarrow$ \textit{Primary download site}\\
	na \textit{http://192.168.137.24/dl}
	\item \textit{Target options} $\rightarrow$ \textit{Target ABI} na \textit{EABI}
	\item \textit{Toolchain} $\rightarrow$ \textit{Toolchain type} na \textit{External toolchain}
	\item \textit{Toolchain} $\rightarrow$ \textit{Toolchain} na \textit{Sourcery CodeBench ARM 2014.05}
\end{enumerate}
Ponadto, zaznaczono opcję
\begin{quote}
	\textit{Filesystem images} $\rightarrow$ \textit{initial RAM filesystem linked into linux kernel}
\end{quote}
oraz włączono kompresję obrazu ustawiając opcję
\begin{quote}
	\textit{Filesystem images} $\rightarrow$ \textit{tar the root filesystem, Compression method (gzip)}
\end{quote}

\subsection{Zadania}
\subsubsection{Aplikacja}
Stworzona w ramach laboratorium aplikacja w języku C jest licznikiem, który wykorzystuje cztery diody LED i trzy przyciski. Diody służą wyświetlaniu binarnej reprezentacji liczby z zakresu $[0,15]$, natomiast przyciski odpowiadają kolejno za dodawanie, odejmowanie i zakończenie obliczeń.\\[\baselineskip]
Dodatkowa część zadania przewidywała dodanie funkcjonalności polegającej na resetowaniu licznika w przypadku wciśnięcia kombinacji dwóch klawiszy.\\[\baselineskip]
Obsługa pinów GPIO jest realizowana przy pomocy interfejsu \texttt{sysfs} dostarczonego z systemem Linux. Program jest w stanie zapalać i gasić diody oraz reagować na wciskanie przycisków poprzez wykonywanie operacji na plikach specjalnych.
\begin{lstlisting}[style=bash, language=c, caption={Funkcja main w pliku counter.c}]
int main(void) {
	int num = 0, work = 1, pressed;
	int btnfds[BTN_COUNT];
	int ledfds[LED_COUNT];
	int btnpins[BTN_COUNT] = { BTN_LEFT, BTN_RIGHT, BTN_TOP };
	int ledpins[LED_COUNT] = { LED_BLUE, LED_WHITE, LED_GREEN, LED_RED };

	export_fds(btnfds, btnpins, "in", BTN_COUNT);
	export_fds(ledfds, ledpins, "out", LED_COUNT);
	setup_edges(btnpins, "falling", BTN_COUNT);

	while (work) {
		light_leds(ledfds, LED_COUNT, num);
		while ((pressed = poll_btns(btnfds, BTN_COUNT)) < 0);
		if (pressed == 3) {
			num = 0;
			continue;
		}
		switch (btnpins[pressed]) {
			case BTN_LEFT:
				num = mod(--num, MAX_NUM);
				break;
			case BTN_RIGHT:
				num = mod(++num, MAX_NUM);
				break;
			case BTN_TOP:
				work = 0;
		}
	}

	unexport_fds(btnfds, btnpins, BTN_COUNT);
	unexport_fds(ledfds, ledpins, LED_COUNT);
	return EXIT_SUCCESS;
}
\end{lstlisting}
Aplikacja rozpoczyna swoje działanie od właściwego skonfigurowania pinów, które odpowiadają obsługiwanym przyciskom i diodom.\\[\baselineskip]
Funkcja \texttt{export\_fds} wpisuje do pliku \texttt{/sys/class/gpio/export} odpowiednie numery pinów, czego efektem jest utworzenie katalogów \texttt{/sys/class/gpio/gpioNN}, gdzie \texttt{NN} jest numerem pinu. Kolejnym krokiem jest ustalenie kierunku na \texttt{in} dla przycisków i \texttt{out} dla diod poprzez wpisanie tych wartości do plików \texttt{direction} odpowiadających każdemu pinowi. Ostatecznie, otwierane są pliki \texttt{value}, a ich deskryptory zapamiętane w tablicach \texttt{btnfds} oraz \texttt{ledfds}.\\[\baselineskip]
Funkcja \texttt{setup\_edges} otwiera pliki \texttt{edge} odpowiadające przyciskom i wpisuje \texttt{falling} do każdego z nich. Dzięki temu aplikacja będzie otrzymywać przerwanie w momencie wciśnięcia przycisku.\\[\baselineskip]
Główna pętla aplikacji zaczyna się od wywołania funkcji \texttt{light\_leds}, która wpisując 0 lub 1 do plików \texttt{value} odpowiednio gasi i zapala diody LED w zależności od obecnej wartości zmiennej \texttt{num}. Następnie, w pętli wywoływana jest funkcja \texttt{poll\_btns}, której zadaniem jest zidentyfikowanie wciśniętego przycisku i zwrócenie jego indeksu, dopóki zwracany przez nią indeks jest mniejszy od zera. Będzie on bowiem wynosił -1, jeżeli zawiedzie debouncing lub zostaną wciśnięte dwa przyciski jednocześnie, z wyjątkiem przycisków odpowiadających dodawaniu i odejmowaniu, co będzie skutkować zresetowaniem licznika. Funkcja \texttt{poll\_btns} wykrywa ich jednoczesne wciśnięcie i zwraca specjalną wartość, która jest sprawdzana przed instrukcją \texttt{switch} w funkcji \texttt{main}.\\[\baselineskip]
Rozpoznawanie wciśniętego przycisku jest realizowane przy pomocy funkcji \texttt{poll}, która reaguje na przerwania generowane przez przyciski. Wykorzystano dwa wywołania tej funkcji - pierwsze z parametrem \texttt{timeout} wynoszącym -1, a drugie z wynoszącym 50. Pierwsze wywołanie będzie czekać na wciśnięcie przycisku, a drugie zakończy oczekiwanie po pięćdziesięciu milisekundach. Jeżeli drugie wywołanie nie zaobserwuje zmiany w stanach przycisków, nastąpi sprawdzenie wartości w plikach \texttt{value} odpowiadających przyciskom, aby zidentyfikować wszystkie wciśnięte. Taka implementacja ma na celu wyleminowanie drgania (\textit{ang. bouncing}) przycisku, które polega na kilkukrotnej zmianie stanu przycisku krótko po jego pierwszym wciśnięciu lub odciśnięciu.\\[\baselineskip]
Przed każdym wywołaniem funkcji \texttt{poll} następuje przeczytanie wartości we wszystkich plikach \texttt{value} odpowiadających przyciskom, po wcześniejszym przesunięciu się na ich początek przy pomocy funkcji \texttt{lseek}. Jest to konieczne, ponieważ wywołania te mogłyby się skończyć natychmiast z powodu wcześniej występujących przerwań.\\[\baselineskip]
Po zidentyfikowaniu wciśniętego przycisku wykonywana jest odpowiednia akcja - dodawanie, odejmowanie lub wyjście z głównej pętli. Ta ostatnia spowoduje, że wywołane zostaną funkcje \texttt{unexport\_fds}, które wpiszą do pliku \texttt{/sys/class/gpio/unexport} numery pinów wyeksportowanych na początku działania programu.

\subsubsection{Pakiet Buildroota}
Aby stworzyć pakiet Buildroota zawierający wyżej opisaną aplikację, do katalogu \texttt{package/} dodano \texttt{gpio-counter/}, w którym utworzono dwa pliki: \texttt{Config.in} oraz \texttt{gpio-counter.mk}.\\[\baselineskip]
Pierwszy z nich ma za zadanie zdefiniowanie pakietu.
\begin{lstlisting}[style=bash, caption={Zawartość pliku Config.in}]
^config^ BR2_PACKAGE_GPIO_COUNTER
	^bool^ "gpio-counter"
	^help^
	  Binary counter using GPIO.

	  https://github.com/felskit
\end{lstlisting}
Drugi plik odpowiada za definicję metody instalacji źródeł, do których podano w nim ścieżkę.
\begin{lstlisting}[style=bash, caption={Zawartość pliku gpio-counter.mk}, keywordstyle=\color{black}]
################################################################################
#
# gpio-counter
#
################################################################################

GPIO_COUNTER_VERSION = 1.0
GPIO_COUNTER_SITE = ^$(TOPDIR)^/../felskit-lab2/gpio-counter
GPIO_COUNTER_SITE_METHOD = local
GPIO_COUNTER_LICENSE = MIT

^define^ GPIO_COUNTER_BUILD_CMDS
   ^$(MAKE)^ ^$(TARGET_CONFIGURE_OPTS)^ all -C ^$(@D)^
^endef^

^define^ GPIO_COUNTER_INSTALL_TARGET_CMDS
    ^$(INSTALL)^ -D -m 0755 ^$(@D)^/counter ^$(TARGET_DIR)^/usr/bin
^endef^

$(eval $(generic-package))
\end{lstlisting}
Tutaj plikiem źródłowym jest \texttt{counter.c}, który zgodnie z instrukcją zawartą w powyższym pliku zostanie skompilowany do pliku \texttt{counter} przy pomocy dostarczonego Makefile'a.
\begin{lstlisting}[style=bash, caption={Zawartość pliku Makefile}]
CC=^$(CROSS_COMPILE)^gcc
CFLAGS=-Wall
all:
	^$(CC)^ ^$(CFLAGS)^ counter.c -o counter
.PHONY: clean
clean:
	-rm counter
\end{lstlisting}
W pliku \texttt{package/Config.in} w sekcji \textit{Debugging, profiling and benchmark} należy dodać linię
\begin{lstlisting}[style=bash]
source "package/gpio-counter/Config.in"
\end{lstlisting}
Wówczas możliwe będzie dodanie naszej paczki poprzez zaznaczenie opcji
\begin{quote}
	\textit{Target packages} $\rightarrow$ \textit{Debugging, profiling and benchmark} $\rightarrow$ \textit{gpio-counter}
\end{quote}

\subsubsection{Korzystanie z debuggera}
Komputer laboratoryjny do zdalnego debugowania potrzebuje działającej na płytce instacji serwera \texttt{gdb}. Wobec tego należy zaznaczyć opcję
\begin{quote}
	\textit{Target packages} $\rightarrow$ \textit{Debugging, profiling and benchmark} $\rightarrow$ \textit{gdb}
\end{quote}
Opcja \textit{gdbserver} będzie domyślnie zaznaczona. Aby komputer mógł się połączyć z działającą instancją serwera i zdalnie debugować zbudowaną aplikację, konieczne jest wybranie opcji
\begin{quote}
	\textit{Toolchain} $\rightarrow$ \textit{Build cross gdb for the host}
\end{quote}
Aby aplikacje stworzone w trakcie kompilacji obrazu systemu mogły być debugowane, konieczne jest włączenie dodawania symboli, co można zrobić zaznaczając opcję
\begin{quote}
	\textit{Build options} $\rightarrow$ \textit{build packages with debugging symbols}
\end{quote}
Pierwszym krokiem do zdalnego debugowania naszej aplikacji jest uruchomienie instacji serwera \texttt{gdb} na płytce przy pomocy polecenia
\begin{lstlisting}[style=bash]
$ gdbserver host:56000 /usr/bin/counter
\end{lstlisting}
Następnie, na stacji roboczej można zestawić połączenie z płytką poleceniami
\begin{lstlisting}[style=bash]
$ /malina/felskit/buildroot-2016.11.2/output/host/usr/bin/arm-linux-gdb \ 
  /malina/felskit/buildroot-2016.11.2/output/build/gpio-counter-1.0/counter
$ (gdb) target remote 192.168.143.188:56000
\end{lstlisting}
Powyższe operacje pozwolą nam na zdalne debugowanie kodu na płytce z komputera laboratoryjnego. Możliwe jest między innymi ustawianie breakpointów przy pomocy polecenia \texttt{break <line>}, kontynuowanie działania programu do kolejnego breakpointu w kodzie poleceniem \texttt{continue} oraz wyświetlanie wartości zmiennej za pomocą polecenia \texttt{display <name>}.
\end{document}
